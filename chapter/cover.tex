%----------------------------------------------------------------------------------------
%   Cover
%----------------------------------------------------------------------------------------
\thispagestyle{empty}
\begin{figure}[ht]
\centering
\includegraphics[height=2.75cm]{images/scutlogo.pdf}
\end{figure}
\begin{center}
\zihao{0}
\textbf{本科毕业论文}
\end{center}
\nopagebreak[4]
\begin{center}
\zihao{1}
\ \\
\end{center}
\nopagebreak[4]
\begin{center}
\zihao{2}
\textbf{你的本科毕业论文题目}
\end{center}
\nopagebreak[4]
\begin{center}
\zihao{1}
\ \\\ \\\ \\
\end{center}
\nopagebreak[4]
\begin{spacing}{1.8}
\begin{center}
 \zihao{-3}
%  \begin{tabular}{c @{ : } c }
%  	\textbf{学 \quad\quad 院} & xxxxxxxx学院 \\ \cline{2-2}
%  	\textbf{专 \quad\quad 业} & xxxxxxx \\ \cline{2-2}
%  	\textbf{学生姓名} & xxx \\ \cline{2-2}
%  	\textbf{学生学号} & xxxxxxxxxx  \\ \cline{2-2}
%  	\textbf{指导老师} & xxx	 \\ \cline{2-2}
%  	\textbf{提交日期} & xxxx 年 x 月 x 日 \\ \cline{2-2}
%  \end{tabular}

% 改用 \makebox 进行分散对齐,参考 https://blog.csdn.net/RobertChenGuangzhi/article/details/50467450 以及 https://zhuanlan.zhihu.com/p/24339981
% 1em 表示当前字体尺寸。4em 即为四个字的大小。如果发现文字溢出,修改 \makebox[xem] 即可

 \begin{tabular}{c @{ : } c }
	\makebox[4em][s]{\textbf{学 \hspace{\fill} 院}} & \makebox[10em][c]{xxxxxxxx学院 } 
		\\ \cline{2-2}
	\makebox[4em][s]{\textbf{专 \hspace{\fill} 业}} & \makebox[10em][c]{xxxxxxx}
		\\ \cline{2-2}
	\makebox[4em][s]{\textbf{学生姓名}}             & \makebox[10em][c]{xxx} 
		\\ \cline{2-2}
	\makebox[4em][s]{\textbf{学生学号}}             & \makebox[10em][c]{xxxxxxxxxx} 
		\\ \cline{2-2}
	\makebox[4em][s]{\textbf{指导老师}}             & \makebox[10em][c]{xxx}	 
		\\ \cline{2-2}
	\makebox[4em][s]{\textbf{提交日期}}             & \makebox[10em][c]{xxxx 年 x 月 x 日} 
		\\ \cline{2-2}
\end{tabular}

\end{center}
\end{spacing}
\pagebreak[4]